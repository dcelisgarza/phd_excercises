\documentclass[10pt,a4paper]{article}
\usepackage{fontspec}
\usepackage{amsmath}
\usepackage{amsfonts}
\usepackage{amssymb}
\usepackage{graphicx}
\usepackage[left=1.00in, right=1.00in, top=1.00in, bottom=1.00in]{geometry}
\usepackage{lmodern}
%\usepackage{minted}
\begin{document}
	\title{Lecture 1}
	\author{Daniel Celis Garza}
	\date{\today}
	\maketitle
	\section{Implicit Methods}
	
	\subsection{Backwards Euler}
	Suppose we have an ODE
	\begin{eqnarray}
		\dfrac{\mathrm{d}y}{\mathrm{d}t} = f(t,y)
	\end{eqnarray}
	Taylor expansion about $t_{n+1}$ where $h$ is the time step,
	\begin{eqnarray}\label{eq:backeul}
		y_{n} &=& y_{n+1} - h f(t_{n+1},y_{n+1}).
	\end{eqnarray}
	
	We need to know what $y_{n+1}$ looks like before using the method. Define an operator, $\mathbf{M}$, acting on $y_{n+1}$,	
	\begin{eqnarray}
		\mathbf{M} y_{n+1} &\equiv& f(t_{n+1}, y_{n+1}) \\
		y_{n+1} &=& y_{n} + h \mathbf{M} y_{n+1} \\
		(1 - \mathbf{M}) y_{n+1} &=& y_{n}.
	\end{eqnarray}
	We can also solve this algebraic equation via fixed-point iteration for the unknown $y_{n+1}$,
	\begin{eqnarray}
		y_{n+1}^{[0]} &=& y_{n} \\
		y_{n+1}^{[i+1]} &=& y_{n} + h f(t_{n+1},y_{n+1}^{[i]})
	\end{eqnarray}
	or a numerical analogue of the Newton-Raphson method.
	
\end{document}